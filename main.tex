\documentclass{tufte-book} % Use the tufte-book class which in turn uses the tufte-common class

\hypersetup{colorlinks} % Comment this line if you don't wish to have colored links

\usepackage{microtype} % Improves character and word spacing

\usepackage{lipsum} % Inserts dummy text

\usepackage{booktabs} % Better horizontal rules in tables

\usepackage{graphicx} % Needed to insert images into the document
\graphicspath{{graphics/}} % Sets the default location of pictures
\setkeys{Gin}{width=\linewidth,totalheight=\textheight,keepaspectratio} % Improves figure scaling

\usepackage{fancyvrb} % Allows customization of verbatim environments
\fvset{fontsize=\normalsize} % The font size of all verbatim text can be changed here

\newcommand{\hangp}[1]{\makebox[0pt][r]{(}#1\makebox[0pt][l]{)}} % New command to create parentheses around text in tables which take up no horizontal space - this improves column spacing
\newcommand{\hangstar}{\makebox[0pt][l]{*}} % New command to create asterisks in tables which take up no horizontal space - this improves column spacing

\usepackage{xspace} % Used for printing a trailing space better than using a tilde (~) using the \xspace command

\newcommand{\monthyear}{\ifcase\month\or January\or February\or March\or April\or May\or June\or July\or August\or September\or October\or November\or December\fi\space\number\year} % A command to print the current month and year

\newcommand{\openepigraph}[2]{ % This block sets up a command for printing an epigraph with 2 arguments - the quote and the author
\begin{fullwidth}
\sffamily\large
\begin{doublespace}
\noindent\allcaps{#1}\\ % The quote
\noindent\allcaps{#2} % The author
\end{doublespace}
\end{fullwidth}
}

\newcommand{\blankpage}{\newpage\hbox{}\thispagestyle{empty}\newpage} % Command to insert a blank page

\usepackage{units} % Used for printing standard units

\newcommand{\hlred}[1]{\textcolor{Maroon}{#1}} % Print text in maroon
\newcommand{\hangleft}[1]{\makebox[0pt][r]{#1}} % Used for printing commands in the index, moves the slash left so the command name aligns with the rest of the text in the index 
\newcommand{\hairsp}{\hspace{1pt}} % Command to print a very short space
\newcommand{\ie}{\textit{i.\hairsp{}e.}\xspace} % Command to print i.e.
\newcommand{\eg}{\textit{e.\hairsp{}g.}\xspace} % Command to print e.g.
\newcommand{\na}{\quad--} % Used in tables for N/A cells
\newcommand{\measure}[3]{#1/#2$\times$\unit[#3]{pc}} % Typesets the font size, leading, and measure in the form of: 10/12x26 pc.
\newcommand{\tuftebs}{\symbol{'134}} % Command to print a backslash in tt type in OT1/T1

\providecommand{\XeLaTeX}{X\lower.5ex\hbox{\kern-0.15em\reflectbox{E}}\kern-0.1em\LaTeX}
\newcommand{\tXeLaTeX}{\XeLaTeX\index{XeLaTeX@\protect\XeLaTeX}} % Command to print the XeLaTeX logo while simultaneously adding the position to the index

\newcommand{\doccmdnoindex}[2][]{\texttt{\tuftebs#2}} % Command to print a command in texttt with a backslash of tt type without inserting the command into the index

\newcommand{\doccmddef}[2][]{\hlred{\texttt{\tuftebs#2}}\label{cmd:#2}\ifthenelse{\isempty{#1}} % Command to define a command in red and add it to the index
{ % If no package is specified, add the command to the index
\index{#2 command@\protect\hangleft{\texttt{\tuftebs}}\texttt{#2}}% Command name
}
{ % If a package is also specified as a second argument, add the command and package to the index
\index{#2 command@\protect\hangleft{\texttt{\tuftebs}}\texttt{#2} (\texttt{#1} package)}% Command name
\index{#1 package@\texttt{#1} package}\index{packages!#1@\texttt{#1}}% Package name
}}

\newcommand{\doccmd}[2][]{% Command to define a command and add it to the index
\texttt{\tuftebs#2}%
\ifthenelse{\isempty{#1}}% If no package is specified, add the command to the index
{%
\index{#2 command@\protect\hangleft{\texttt{\tuftebs}}\texttt{#2}}% Command name
}
{%
\index{#2 command@\protect\hangleft{\texttt{\tuftebs}}\texttt{#2} (\texttt{#1} package)}% Command name
\index{#1 package@\texttt{#1} package}\index{packages!#1@\texttt{#1}}% Package name
}}

% A bunch of new commands to print commands, arguments, environments, classes, etc within the text using the correct formatting
\newcommand{\docopt}[1]{\ensuremath{\langle}\textrm{\textit{#1}}\ensuremath{\rangle}}
\newcommand{\docarg}[1]{\textrm{\textit{#1}}}
\newenvironment{docspec}{\begin{quotation}\ttfamily\parskip0pt\parindent0pt\ignorespaces}{\end{quotation}}
\newcommand{\docenv}[1]{\texttt{#1}\index{#1 environment@\texttt{#1} environment}\index{environments!#1@\texttt{#1}}}
\newcommand{\docenvdef}[1]{\hlred{\texttt{#1}}\label{env:#1}\index{#1 environment@\texttt{#1} environment}\index{environments!#1@\texttt{#1}}}
\newcommand{\docpkg}[1]{\texttt{#1}\index{#1 package@\texttt{#1} package}\index{packages!#1@\texttt{#1}}}
\newcommand{\doccls}[1]{\texttt{#1}}
\newcommand{\docclsopt}[1]{\texttt{#1}\index{#1 class option@\texttt{#1} class option}\index{class options!#1@\texttt{#1}}}
\newcommand{\docclsoptdef}[1]{\hlred{\texttt{#1}}\label{clsopt:#1}\index{#1 class option@\texttt{#1} class option}\index{class options!#1@\texttt{#1}}}
\newcommand{\docmsg}[2]{\bigskip\begin{fullwidth}\noindent\ttfamily#1\end{fullwidth}\medskip\par\noindent#2}
\newcommand{\docfilehook}[2]{\texttt{#1}\index{file hooks!#2}\index{#1@\texttt{#1}}}
\newcommand{\doccounter}[1]{\texttt{#1}\index{#1 counter@\texttt{#1} counter}}

\usepackage{makeidx} % Used to generate the index
\makeindex % Generate the index which is printed at the end of the document

% This block contains a number of shortcuts used throughout the book
\newcommand{\vdqi}{\textit{VDQI}\xspace}
\newcommand{\ei}{\textit{EI}\xspace}
\newcommand{\ve}{\textit{VE}\xspace}
\newcommand{\be}{\textit{BE}\xspace}
\newcommand{\VDQI}{\textit{The Visual Display of Quantitative Information}\xspace}
\newcommand{\EI}{\textit{Envisioning Information}\xspace}
\newcommand{\VE}{\textit{Visual Explanations}\xspace}
\newcommand{\BE}{\textit{Beautiful Evidence}\xspace}
\newcommand{\TL}{Tufte-\LaTeX\xspace}

%----------------------------------------------------------------------------------------
%	BOOK META-INFORMATION
%----------------------------------------------------------------------------------------

\title{Graphs and Information Retrieval\thanks{Thanks to Edward R.~Tufte for his inspiration.}} % Title of the book

\author[Chris Kamphuis]{Chris Kamphuis} % Author

\publisher{Publisher of This Book} % Publisher

%----------------------------------------------------------------------------------------
\def\mytitle{Graphs and Information Retrieval}
\def\myauthor{Chris Frans Henri Kamphuis}
\def\mydate{\formatdate{1}{6}{2023}}

% titlepage

\begin{document}
\begin{titlepage}
	\begin{fullwidth}
	\begin{center}
		\vspace*{3.5cm}
		
		%		\LARGE{\textsc{\bfseries\mytitle}}
		\huge{\bfseries\mytitle}
		
		\vspace*{15pt}
		
		%		\large{\textsc{\mysubtitle}}
		% \Large{\mysubtitle}
		
		\vspace*{5pt}
		
		%	    \large{\textsc{More subtitle}}
		\normalsize
		
		\vspace{2.0cm}
		
		Proefschrift
		
		\vspace{0.5cm}
		
		ter verkrijging van de graad van doctor\\
		aan de Radboud Universiteit Nijmegen\\
		op gezag van de rector magnificus prof.~dr.~J.H.J.M.\ van\ Krieken,\\
		volgens besluit van het college van decanen\\
		in het openbaar te verdedigen
		
		\vspace{0.5cm}
		
		op woensdag 22 maart 2023 \\
		om 12:00 uur precies
		
		\vspace{0.5cm}
		
		door
		
		\vspace{0.5cm}
		
		\myauthor\\
		
		geboren op 22 maart 1993 te Oldenzaal, Nederland
	\end{center}
	\end{fullwidth}
\end{titlepage}

\newpage%

% copyright page

\begin{fullwidth}
	\begin{itemize}[leftmargin=*]
		\item[] Promotor:
		\begin{itemize}
			\item[] prof.\ dr.\ ir.\ A. P.\ (Arjen)\ de Vries
		\end{itemize}
	\end{itemize}
	
	\begin{itemize}[leftmargin=*]
		\item[] Manuscriptcommissie:
		\begin{itemize}
			\item[] \makebox[4.2cm]{Person A\hfill} (Affiliation)
			\item[] \makebox[4.2cm]{Person B\hfill} (Affiliation)
			\item[] \makebox[4.2cm]{Person C\hfill} (Affiliation)
		\end{itemize}
	\end{itemize}
	~\vfill
	\thispagestyle{empty}
	\setlength{\parindent}{0pt}
	\setlength{\parskip}{\baselineskip}
	
	\par This work is part of the research program Commit2Data with project number 628.011.001 (SQIREL-GRAPHS), which is (partly) financed by the Netherlands Organisation for Scientific Research (NWO).

	
	Copyright \copyright\ \the\year\ \thanklessauthor
	
	\par\smallcaps{Published by \thanklesspublisher}
	
	\par\smallcaps{tufte-latex.googlecode.com}
	
	\par Licensed under the Apache License, Version 2.0 (the ``License''); you may not use this file except in compliance with the License. You may obtain a copy of the License at \url{http://www.apache.org/licenses/LICENSE-2.0}. Unless required by applicable law or agreed to in writing, software distributed under the License is distributed on an \smallcaps{``AS IS'' BASIS, WITHOUT WARRANTIES OR CONDITIONS OF ANY KIND}, either express or implied. See the License for the specific language governing permissions and limitations under the License.\index{license}
	
	\par\textit{First printing, \monthyear}
\end{fullwidth}

\frontmatter

%----------------------------------------------------------------------------------------
%	DEDICATION PAGE
%----------------------------------------------------------------------------------------

\cleardoublepage
~\vfill
\begin{doublespace}
	\noindent\fontsize{18}{22}\selectfont\itshape
	\nohyphenation
	Dedicated to Maudy
\end{doublespace}
\vfill
\vfill


\tableofcontents % Print the table of contents

%----------------------------------------------------------------------------------------
% here or at back or not
\listoffigures % Print a list of figures

%----------------------------------------------------------------------------------------
% here or at back or not
\listoftables % Print a list of tables

%----------------------------------------------------------------------------------------
%	EPIGRAPH
%----------------------------------------------------------------------------------------
\newpage
\thispagestyle{empty}
\hspace{0pt}
\vfill
\begin{center}
	\openepigraph{I PROPOSE to consider the question, ``Can machines think?''}{Alan Turing - 1950}
\end{center}
\vfill
\hspace{0pt}


\mainmatter

%----------------------------------------------------------------------------------------
%	INTRODUCTION
%----------------------------------------------------------------------------------------

\cleardoublepage
\chapter{Introduction}

I also propose to consider the question, ``Can machines think?'' Instead of approaching this question through a thought experiment like Turing\cite{Turing-Think} did, nowadays one can approach this question by prompting it to a search engine. 

When prompting this question to popular web search engines we get different results: 
%----------------------------------------------------------------------------------------

%----------------------------------------------------------------------------------------
%	CHAPTER 1
%----------------------------------------------------------------------------------------

\chapter{Information Retrieval using Relational Databases}
\label{ch:tufte-design}

\newthought{The pages} of a book are usually divided into three major sections: the front matter (also called preliminary matter or prelim), the main matter (the core text of the book), and the back matter (or end matter). Which is about information retrieval \index{Information Retrieval}


\newpage
\thispagestyle{empty}
\hspace{0pt}
\vfill
\begin{center}
	\openepigraph{We can only see a short distance ahead, but we can see plenty there that needs to be done.}{Alan Turing}
\end{center}
\vfill
\hspace{0pt}

%----------------------------------------------------------------------------------------
%	CHAPTER 2
%----------------------------------------------------------------------------------------


\chapter[Conclusion]{Conclusion}
\label{ch:conclusion}

We finalize 
%------------------------------------------------

%----------------------------------------------------------------------------------------

\backmatter

%----------------------------------------------------------------------------------------
%	BIBLIOGRAPHY
%----------------------------------------------------------------------------------------

\bibliography{bibliography} % Use the bibliography.bib file for the bibliography
\bibliographystyle{plainnat} % Use the plainnat style of referencing

%----------------------------------------------------------------------------------------

\chapter*{Summary}
In thi

\chapter*{Samenvatting}
In thi

\chapter*{Acknowledgements}
In thi

\chapter*{Research Data Management}
In thi

\chapter*{Curriculum Vitæ}
In thi





\printindex % Print the index at the very end of the document

\end{document}