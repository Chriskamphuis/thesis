\chapter{Introduction}

% In this chapter we first introduce the fields of Information Retrieval and Databases, IR focusses on the data management and processing for search engines, whereas Database Research investigates the processing of general structured data. Information Retrieval researchers often use their own data structures, which are specialized for IR applications. We compare this to what is often done at database research, and show examples where these fields overlap. 

I also propose to consider the question, ``Can machines think?'' Instead of approaching this through a thought experiment like Turing did, nowadays one can approach this question by asking it to a search engine. When issuing this query to popular web search engines we get different results; the first result on Google is a passage generated from the article written by Turing, while the first result on Bing is a passage generated from a website that states machines can not think\footnote{However, if a machine can not think, can we trust the result presented by this algorithm?}.
We use these systems that process queries every day in our lives to provide us information. Whereas Google and Bing are all purpose web engines that mainly focus on finding and retrieving web data, people also used specialized search systems in their day-to-day lives, examples are: Amazon / EBay for product search, NS for public transport in the Netherlands, Scholar / Zeta Alpha for scientific resources, Youtube / TikTok for Videos, or Facebook / LinkedIn for people.
When searching for the query ``Can machines think?'', the approach of searching through text document only might be sufficient for the user. However in many cases when searching today, only considering text is not sufficient. When one wants to buy a product on Amazon, aspects other than text also need to be considered. Lets say for example you want to buy an IPhone; What is the price, which edition is the most recent, or which color does it have. When someone searches for people on LinkedIn, they are generally more interested in persons that have connections in common compared to complete strangers. If you are looking for someone to do a job, it is ideal that a shared connection can vouch for them. 

\section{Information Retrieval}
Everything that is needed to process a query like, ``Can machines think?'', is subject to research by the field of information retrieval. 

\subsection{Inverted Indexes}

\section{Relational Databases}
Relational databases are usually used to store structure data. 

\section{Graphs}
Instead of using columnar data, it might be more attractive to model your data using graphs. 