\chapter{Introduction}
\epigraph{I \textit{propose} to consider the question, ``Can machines think?''}{Alan Turing - 1950}

\section{Background and Context}
I also propose to consider the question, ``Can machines think?'' Instead of approaching this through a thought experiment like Turing did, nowadays one can approach this question by asking it to a search engine. When issuing this query to popular web search engines we get varying results: the first result on Google is a passage generated from the article written by Turing, while the first result on Bing is a passage generated from a website that concludes machines can not think\footnote{However, if a machine can not think, can we trust the result presented by this algorithm?}.

We use systems that process queries every day in our lives when we are looking for \textit{information}. While Google and Bing are all purpose web engines that mainly focus on finding and retrieving information from the internet, people also used specialized search systems in their day-to-day lives: Amazon and EBay for product search, Scholar and ResearchGate for scientific resources, Youtube and TikTok for Videos, or Facebook and LinkedIn when we are searching for people. It might even be possible that you are reading this text, after you found this document through search. 

When searching for the query ``Can machines think?'', the approach of searching through text documents only, might be sufficient for the person who searches. However in many cases when searching today, only considering text is not sufficient. For example, when one wants to buy a product on Amazon, aspects other than text also need to be considered. Lets say for example you want to buy an IPhone; What is the price, which edition is the most recent, or which color does it have.

When someone searches for people on LinkedIn, they are generally more interested in persons that have connections in common compared to complete strangers. If you are looking for someone to do a job, it is ideal that a shared connection can vouch for them. In this case, how people relate to each other in their network might be an indication of \textit{relevance}. Not only the structure on how people relate to each other determines relevance; other examples are their experience, where they work, or reviews on their previous work might matter.

Although it might be possible to encode all this information in written form, often it is more convenient to save this information in a more structured approach. In this work 

\section{Problem Description and Research Questions}


\section{Thesis Contributions and Structure}



\section{Publications}